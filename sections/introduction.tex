\section{The Deep Underground Neutrino Experiment}
The \textbf{Deep Underground Neutrino Experiment (DUNE)} is a neutrino experiment currently under construction comprised of a near detector at Fermilab and a far detector at the Sanford Underground Research Facility (SURF).
A neutrino beam produced at Fermilab will propagate over a distance if 1300 km through the Earth, and emerge at the SURF laboratory in South Dakota where they will be detected by a 40-kiloton liquid argon time projection chamber (LArTPC).
This experimental setup allows many aspects of neutrino physics to be probed.
These are the overarching physics goals which can be listed as follows.

\begin{itemize}
\item \textbf{Understanding of the matter-antimatter asymmetry in the Universe.} After the big bang, it is assumed that matter and antimatter were produced in equal quantities, however, we observe a matter dominated Universe with comparatively negligible quantities of antimatter. By studying neutrino oscillations we can determine the amount by which the charge-parity (CP) symmetry of the universe is violated in the lepton sector. 

\item \textbf{Determining the underlying symmetries of the Universe.} The links between particle masses and mixings are not fully understood by the Standard Model. DUNE aims to make precise measurements of the neutrino mixing parameters and determine the neutrino mass hierarchy. By comparing these results to those found in the quark sector, new underlying symmetries of the Universe could be uncovered.

\item \textbf{Determining whether there is a Grand Unified Theory (GUT) of the Universe.} Numerous unifications of forces have been discovered in recent history. First Maxwell theorised the unification of the electric and magnetic forces into the electromagnetic force \cite{maxwell1873treatise}. Later Glashow, Salam and Weinberg presented the unification of the electromagnetic force with the weak nuclear force, resulting in the electroweak force \cite{glashow1959renormalizability, salam1959weak, weinberg1967model}. Current GUTs that attempt to unify all of the fundamental forces together predict proton decay, a process that to this day has not been observed. DUNE aims to search for proton decay over a range of potential lifetimes predicted by these GUTs.

\item \textbf{How do supernovae explode and what can we learn from their associated neutrino bursts.} Only a finite number of the elements we know of can be produced in the main phase of a star's life cycle. Heavier elements come from supernova explosions where the super-hot cores of massive stars collapse in on themselves. DUNE will have the capability to detect the neutrino burst from a core-collapse supernova occurring within our galaxy. The timing, energy and flavour structure of the neutrino burst will offer information into the dynamics of supernovae, as well as additional information on neutrino properties.

\end{itemize} \cite{collaboration2016long}
\subsection{Physics programs for DUNE}
The physics programs at DUNE can be split into two separate programs, each of which has their specific goals.
The primary program consists of:
\begin{itemize}
\item making precise measurements of the neutrino oscillation parameters associated with $\nu_{\mu} \rightarrow \nu_{e}$ and $\overline{\nu}_{\mu} \rightarrow \overline{\nu}_{e}$ for the purpose of
\begin{itemize}
\item measuring the CP violating phase $\delta_{\textrm{CP}}$
\item determination of the neutrino mass hierarchy my deducing the sign of $\Delta m^{2}_{31} = m^{2}_{3} - m^{2}_{1}$
\item determination of the octant within which the $\theta_{23}$ mixing angle lies;
\end{itemize}

\item search for proton decay through the $p \rightarrow K^{+}\overline{\nu}$ channel, simultaneously showing the violation of baryon and lepton number conservation;  

\item detection of $\nu_{e}$ flux from a core-collapse supernova should one occur within our galaxy during the active lifetime of DUNE.
\end{itemize}

Alongside the primary program there is also the ancillary program which aims to probe the more subtle areas of neutrino properties, such as:
\begin{itemize}
\item Beyond the Standard Model {BSM} physics including sterile neutrinos and tau neutrino appearance;
\item atmospheric neutrino oscillation;
\item and searches for dark matter signatures.

\end{itemize}
  
\noindent The main focus of this report will be on supernova neutrino burst and low-energy neutrino physics studies. 
This will include the theoretical overview and current work being done on the topic.
Additionally, the studies into the backgrounds one can expect from these studies will be discussed in detail as background mitigation is highly important in low-energy studies.


\section{Supernova and low-energy neutrinos}
The DUNE detector will have a particular sensitivity for low energy electron neutrinos, such as those emerging from core-collapse supernovae.
The energy of these neutrinos is typically in the tens of MeV range which would produce short electron tracks within the detector, as well as some additional gamma rays.
With high enough statistics and DUNE's sensitivity the different stages of a core-collapse supernova and be examined, resulting in new understanding in the overall mechanism.

\subsection{Stages of a core-collapse supernova}
Stars produce energy through the nuclear fusion of elements in its core.
Particularly massive stars over time develop a layered structure in their core of the different elements as they are produced. 
In the centre will be iron, with progressively lighter elemental layers surrounding it. 
Eventually, as the fuel the star is using to create the heavier elements depletes, fusion ceases and the core collapses under its own gravity, resulting in a core-collapse supernova.

Core Collapse continues for roughly one hundredth of a second, and then stops.
At this point it is extremely dense at the nucleus of the collapse ($\rho \sim 10^{12} - 10^{14}$) g/cm$^{3}$, creating a material opaque to neutrinos.
Interestingly the temperature of the core at this point is $< 30$ MeV, which is relatively cold.
The gravitational energy of the collapse is stored in a sea of electrons and electron neutrinos, trapped within the region of the collapse.
























































