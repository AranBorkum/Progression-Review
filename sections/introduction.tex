\section{The Deep Underground Neutrino Experiment}
The \textbf{Deep Underground Neutrino Experiment (DUNE)} is a neutrino experiment currently under construction comprised of a near detector at Fermilab and a far detector at the Sanford Underground Research Facility (SURF).
A neutrino beam produced at Fermilab will propagate over a distance if 1300 km through the Earth, and emerge at the SURF laboratory in South Dakota where they will be detected by a 40-kiloton liquid argon time projection chamber (LArTPC).
This experimental setup allows many aspects of neutrino physics to be probed.
These are the overarching physics goals which can be listed as follows.

\begin{itemize}
\item \textbf{Understanding of the matter-antimatter asymmetry in the Universe.} After the big bang, it is assumed that matter and antimatter were produced in equal quantities, however, we observe a matter dominated Universe with comparatively negligible quantities of antimatter. By studying neutrino oscillations we can determine the amount by which the charge-parity (CP) symmetry of the universe is violated in the lepton sector. 

\item \textbf{Determining the underlying symmetries of the Universe.} The links between particle masses and mixings are not fully understood by the Standard Model. DUNE aims to make precise measurements of the neutrino mixing parameters and determine the neutrino mass hierarchy. By comparing these results to those found in the quark sector, new underlying symmetries of the Universe could be uncovered.

\item \textbf{Determining whether there is a Grand Unified Theory (GUT) of the Universe.} Numerous unifications of forces have been discovered in recent history. First Maxwell theorised the unification of the electric and magnetic forces into the electromagnetic force \cite{maxwell1873treatise}. Later Glashow, Salam and Weinberg presented the unification of the electromagnetic force with the weak nuclear force, resulting in the electroweak force \cite{glashow1959renormalizability, salam1959weak, weinberg1967model}. Current GUTs that attempt to unify all of the fundamental forces together predict proton decay, a process that to this day has not been observed. DUNE aims to search for proton decay over a range of potential lifetimes predicted by these GUTs.

\item \textbf{How do supernovae explode and what can we learn from their associated neutrino bursts.} Only a finite number of the elements we know of can be produced in the main phase of a star's life cycle. Heavier elements come from supernova explosions where the super-hot cores of massive stars collapse in on themselves. DUNE will have the capability to detect the neutrino burst from a core-collapse supernova occurring within our galaxy. The timing, energy and flavour structure of the neutrino burst will offer information into the dynamics of supernovae, as well as additional information on neutrino properties.

\end{itemize} \cite{collaboration2016long}
\subsection{Physics programs for DUNE}
The physics programs at DUNE can be split into two separate programs, each of which has their specific goals.
The primary program consists of:
\begin{itemize}
\item making precise measurements of the neutrino oscillation parameters associated with $\nu_{\mu} \rightarrow \nu_{e}$ and $\overline{\nu}_{\mu} \rightarrow \overline{\nu}_{e}$ for the purpose of
\begin{itemize}
\item measuring the CP violating phase $\delta_{\textrm{CP}}$
\item determination of the neutrino mass hierarchy my deducing the sign of $\Delta m^{2}_{31} = m^{2}_{3} - m^{2}_{1}$
\item determination of the octant within which the $\theta_{23}$ mixing angle lies;
\end{itemize}

\item search for proton decay through the $p \rightarrow K^{+}\overline{\nu}$ channel, simultaneously showing the violation of baryon and lepton number conservation;  

\item detection of $\nu_{e}$ flux from a core-collapse supernova should one occur within our galaxy during the active lifetime of DUNE.
\end{itemize}

Alongside the primary program there is also the ancillary program which aims to probe the more subtle areas of neutrino properties, such as:
\begin{itemize}
\item Beyond the Standard Model {BSM} physics including sterile neutrinos and tau neutrino appearance;
\item atmospheric neutrino oscillation;
\item and searches for dark matter signatures.

\end{itemize}
  
\noindent The main focus of this report will be on supernova neutrino burst and low-energy neutrino physics studies. 
This will include the theoretical overview and current work being done on the topic.
Additionally, the studies into the backgrounds one can expect from these studies will be discussed in detail as background mitigation is highly important in low-energy studies.


\section{Supernova and low-energy neutrinos}
The DUNE detector will have a particular sensitivity for low energy electron neutrinos, such as those emerging from core-collapse supernovae.
The energy of these neutrinos is typically in the tens of MeV range which would produce short electron tracks within the detector, as well as some additional gamma rays.
With high enough statistics and DUNE's sensitivity the different stages of a core-collapse supernova and be examined, resulting in new understanding in the overall mechanism.

\subsection{Stages of a core-collapse supernova}
Stars produce energy through the nuclear fusion of elements in its core.
Particularly massive stars over time develop a layered structure in their core of the different elements as they are produced. 
In the centre will be iron, with progressively lighter elemental layers surrounding it. 
Eventually, as the fuel the star is using to create the heavier elements depletes, fusion ceases and the core collapses under its own gravity, resulting in a core-collapse supernova.

Core Collapse continues for about one-hundredth of a second and then stops.
At this point it is extremely dense at the nucleus of the collapse ($\rho \sim 10^{12} - 10^{14}$ g/cm$^{3}$), creating a material opaque to neutrinos.
Interestingly the temperature of the core at this point is $< 30$ MeV, which is relatively cold.
The gravitational energy of the collapse is stored in a sea of electrons and electron neutrinos, trapped within the region of the collapse.
At this point the lepton number of the core is completely confined.

Eventually, the energy stored by the electrons and neutrinos, and consequently the core's lepton number, escape, propagated by the least interacting particles; in this case, the neutrinos as described by the Standard Model.
Energies of roughly $10^{53}$ ergs are released in around ten seconds by $10^{58}$ neutrinos and antineutrinos.
All flavours of neutrinos are produced carrying energies of $\sim 10$ MeV.
Of this energy released a small amount is absorbed with results in the shockwave that blasts away the rest of the star's matter if a huge explosion. 
About 10\% of the star's rest mass is carried away as neutrinos and the central object that remains settles into a neutron star or depending on the star's original mass, a black hole.

\subsection{Core-collapse supernova observables}
The core-collapse can be divided into three separate phases, each spanning a relatively predictable timeframe. 
These three phases all carry different signatures which can be differentiated from one another and used to study the process of the core-collapse.

The first phase of the core-collapse is a short, sharp \textit{neutronisation} burst, composed primarily of $\nu_{e}$.
This burst is followed shortly after by the \textit{accretion} phase, lasting for a a few hundred milliseconds. 
The time span of the accretion phase is largely dependent on the mass of the progenitor star.
As matter falls onto the collapsed core, the shockwave that would be produced becomes stalled at a distance of $\sim 200$ km.
Lastly, the \textit{cooling} phase begins after the accretion.
The cooling phase is where the proto-neutron star expels all of its trapped energy
This lasts for around ten seconds and represents the main component of the observable signal.

The main features of the three phases discussed above are illustrated in Figure \ref{fig:core-collapse-phases}, taken from \cite{WURM2012685}.

\begin{figure}[h] %  figure placement: here, top, bottom, or page
   \centering
   \includegraphics[width=\textwidth]{figures/Supernove-phases.png} 
   \caption{Expected core-collapse neutrino symbol for a $10.8\ M_{\odot}$ progenitor star. The left most plots show the early signal, known as the neutronisation burst; the central two plots represent the accretion phase, and the right most plots show the cooling phase. All of the upper plots are made as luminosity as a function of time, where as the lower plots are the average energy as a function of time. These plots assume that the fluxes of $\nu_{\mu}$, $\overline{\nu}_{\mu}$, $\nu_{\tau}$ and $\overline{\nu}_{\tau}$ are the same.}
   \label{fig:core-collapse-phases}
\end{figure}

\subsection{Detection channels in liquid Argon}
As mentioned previously, DUNE has a particular sensitivity to electron neutrinos emerging from a supernova.
The detection channel in this case is the charged-current (CC) interaction of $\nu_{e}$ on $^{40}$Ar, 
\begin{equation}
\nu_{e}\ +\ ^{40}\textrm{Ar}\ \rightarrow\ e^{-}\ +\ ^{40}\textrm{K}^{*}, 
\end{equation}
\noindent where the dominant observables are the trail the $e_{-}$ leaves in the time projection chamber and the de-excitation products of the excited $K^{*}$.
There are also some subdominant signals, such as $\overline{\nu}_{e}$ interactions and elastic scattering on electrons.
The cross-sections of some of the more relevant interactions within the LArTPC are shown in Figure \ref{fig:SN-cross-sections}

\begin{figure}[h] %  figure placement: here, top, bottom, or page
   \centering
   \includegraphics[width=0.8\textwidth]{figures/SN-cross-sections.png} 
   \caption{Cross-sections as a function of neutrino energy for some of the more supernova-relevant neutrino interactions, figure courtesy of \cite{gil2003oscillation}}
   \label{fig:SN-cross-sections}
\end{figure}

\noindent Along with the dominant $\nu_{e}$ charged-current interaction, neutrino neutral-current interactions are of interest to supernova detection in the LArTPC.
The neutral-current scattering is given as
\begin{equation}
\nu_{x}\ + \textrm{Ar}\ \rightarrow\ \nu_{x}\ +\ \textrm{Ar}^{*}
\end{equation}
\noindent where the main observable is the de-excitation $\gamma$-cascade.
There is a dominant 9.8 MeV decay line for the excited Ar$^{*}$ which has a reasonably high probability of inducing electron pair production, offering a potential neutral-current tag in the future.

\section{Radiological backgrounds at DUNE}

One must carefully consider the backgrounds in any experiment and that is no different for DUNE. 
In the case of these low-energy, MeV scale neutrinos there are numerous backgrounds coming from intrinsic parts of the experimental design.
These backgrounds include:
\begin{itemize}
\item Ionising particles emitted during the U/Th chain coming from the concrete around the detector and the rock surrounding the cavern;

\item Ionising particles coming from naturally occurring Radon in the air;

\item decay products of $^{42}$Ar and $^{39}$Ar in the liquid argon volume;

\item Neutrons of various energies coming from the surrounding rock and concrete,
\end{itemize}

\noindent as well as various other backgrounds, all of which must be carefully dealt with.
It is the neutron backgrounds that will be discussed in more detail as they pertain to the work described in this report.

Neutrons present as a dominant background in low energy neutrino detection.
They are produced in $(\alpha, n)$ reactions as a result of the daughter decays of $^{238}$U and $^{232}$Th, as well as sub-dominantly from the spontaneous fission of $^{238}$U.

It is unlikely for a non-thermal neutron to capture on Argon, and so neutrons tend to elastically scatter around within the volume of liquid Argon.  
They can travel for a few meters, after which they will either escape the detector or capture on the Argon.
There are two very specific energies associated with neutron capture on argon.
These are 6.1 MeV for capture on $^{40}$Ar and 8.8 MeV for capture on $^{37}$Ar.
These energies refer to the sum of the energies of the photon emissions during the de-excitation of $^{40}$Ar$^{*}$ and $^{37}$Ar$^{*}$ respectively.
The energy levels of the excited Argon atoms after neutron capture if shown in Figure \ref{fig:neutron-gamma}

\begin{figure}[h] %  figure placement: here, top, bottom, or page
   \centering
   \includegraphics[width=0.8\textwidth]{figures/n-cap-on-ar.png} 
   \caption{Energy levels of $^{40}$Ar (left) and $^{37}$Ar (right) after capture of a thermal neutron \cite{hardell1970thermal}.}
   \label{fig:neutron-gamma}
\end{figure}

\noindent These photons can then Compton scatter and pair produce as one would expect. 
The difficulty with mitigating neutron backgrounds is that one cannot simple apply a fiducial volume cut on the detector because of how far neutron travel in liquid Argon.
Instead shielding based approaches must be explored in order to lower the background rate.














